\documentclass[a4paper,10pt]{article}

\usepackage[BoldFont,SlantFont,CJKsetspaces,CJKchecksingle]{xeCJK}
\usepackage{fancyhdr}
\usepackage{amsmath}
\usepackage{amssymb}
\usepackage{amsthm}
\usepackage{multirow}
\usepackage{longtable}
\usepackage{graphicx}
\usepackage{xcolor}
\usepackage{float}
\usepackage{tikz,pgfplots}
\usepackage{pifont}
\usetikzlibrary{arrows,scopes,svg.path,shapes,shadows,positioning,decorations.markings}
\usepackage{verbatim}
\usepackage[colorlinks,linkcolor=black,anchorcolor=blue,citecolor=green,filecolor=blue,urlcolor=blue]{hyperref}
\usepackage{listings}
\usepackage{setspace}
\usepackage{indentfirst}
\usepackage{cleveref}
\usepackage{mathrsfs}

\parindent 2em

\setCJKmainfont[BoldFont=SimHei,ItalicFont=KaiTi_GB2312]{SimSun}
\setCJKmonofont{SimSun}

\newtheorem{lemma}{\textbf{引理}}
\newtheorem{theorem}{\textbf{定理}}


\title{Sudoku Game Report}

\author{计64~~陶东来~~~学号:2016011322}

\begin{document}
    \maketitle
    \section{概述}

    本次大作业,我本来想使用传统的Qt来完成。但我在实现过程中发现,我难以简洁地达成我想要的效果。并且,Qt的QSS并不像CSS那样有
    方便的工具(jQuery)来辅助处理。因此我选择转换思路,把工程分解成了由html+css+js管理的前端,和用Qt管理的后端,通过QWebEngine和
    QWebChannel来进行前后端的交互。

    必须说明的是,本次所实现的版本仅仅是一个“原型”。更多的视觉效果,如动画,可以在前端中较为方便地扩展出来(使用keyframe),比起单纯
    的Qt要简单得多。这也是我使用这种设计的重要原因之一。

    \section{前端}

    前端使用了jQuery和bootstrap来辅助实现,通过js来管理游戏的整体运行时逻辑。

    \section{后端}

    后端的Qt主要负责数独盘面的产生,并通过QWebChannel传输至前端。并且,通过Qt后端,游戏进度的保存也成为可能。

    \section{数独的求解及生成}

    数独的求解使用了经典的DLX(Dancing Link X)算法。数独的生成则参考了同学交流时所提到的Las Vegas+贪心挖洞算法。
    而难度则直观地表现在盘面空格的个数上。难度大于等于$x$时,盘面上的空格数应至少为$10+5x$个。

    \section{值得加入的改进}

    通过加入动画让UI更为炫酷;

    保存最佳成绩;

    等等。

\end{document}
